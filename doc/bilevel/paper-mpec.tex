
\lstset{language=Python}
\definecolor{light-gray}{gray}{0.9}
\lstset{backgroundcolor=\color{light-gray}}
\lstset{aboveskip=1em,belowskip=1em,showspaces=false,showstringspaces=false}


\SANDsection{Introduction}

Mathematical Programs with Equilibrium Constraint (MPEC) problems
arise in a large number of applications in engineering and economic
systems~\cite{FerPan97,LouPanRal96,OutKocZow98}.  An MPEC is an
optimization problem that includes equilibrium contraints in the
form of complementarity conditions.  Equilibrium constraints naturally
arise as the solution to an optimization subproblem (e.g. for bilevel
programs), variational inequalities, and
complementarity problems~\cite{HarPan90}.

Since MPEC problems frequently arise in practice, many algebraic
modeling languages (AML) have integrated capabilities for expressing
complementarity conditions~\cite{Mun00}, including AMLs like\linebreak
AIMMS~\citep{AIMMS}, AMPL~\citep{AMPL,FouGay03}, GAMS~\citep{GAMS},
MATLAB~\cite{MATLAB} and YALMIP~\cite{Lof04}.  AMLs are high-level
programming languages for describing
and solving mathematical problems, particularly optimization-related
problems~\citep{Kal04}.  AMLs provide a mechanism for defining
variables and generating constraints with a concise mathematical
representation, which is essential for large-scale, real-world
problems that involve thousands or millions of constraints and
variables.

In this paper, we describe new functionality in Pyomo 4.1 for expressing and
optimizing MPEC models in the Pyomo modeling environment.  Pyomo is an
open-source software package that supports
the definition and solution of optimization applications using the
Python language~\cite{pyomotrac,pyomoweb,HarWatWoo11,HarLaiWatWoo12}.  Python is a powerful 
programming language that has a clear, readable syntax and
intuitive object orientation.  Pyomo uses an object-oriented approach
for defining models that contain decision variables, objectives,
and constraints.  MPEC models can be easily expressed with Pyomo
modeling components for complementarity conditions.  Further, Pyomo's
object-oriented design naturally supports the ability to automate
the reformulation of MPEC models into other forms (e.g. disjunctive
programs).  We describe Pyomo meta-solvers that transform MPECs as
MIP or NLP problems, which are then optimized with standard solvers.  Further,
we describe interfaces to specialized mixed complementarity problem
solvers, which
solve MPEC problems expressed without an optimization objective.

The remainder of this paper is organized as follows.
Section~\ref{sec:design} describes how Pyomo supports modeling of
equilibrium constraints as mixed-complementarity conditions.
Section~\ref{sec:transformations} describes transformation capabilities
that automate the reformulation of MPEC models.  Section~\ref{sec:solvers}
describes meta-solvers in Pyomo that leverage these transformations
to support global and local optimization of MPEC problems.
Section~\ref{sec:discussion}
describes future work that is planned for Pyomo.



\SANDsection{Modeling Equilibrium Conditions}
\label{sec:design}

\subsection{Complementarity Conditions}

\citet{FerFouGay99} note that there are a few fundamental forms
that account for a wide range of complementarity conditions that
arise in practice.  Consider a variable $x$ and function $g(x)$.  The classical form of 
complementarity condition can be expressed as
\[
x \geq 0  \;\bot\;  g(x) \geq 0, 
\]
which expresses the complementarity restriction that at least one of these 
must hold with equality.
When the variable $x$ is bounded such that $x \in [l, u]$, then 
a \textit{mixed complementarity condition} can be expressed as
\[
l \leq x \leq u  \;\bot\;  g(x),
\]
which expresses the complementarity restriction that at least one of the following
must hold:
\[
\begin{array}{llll}
& x = l & \text{and} & g(x) \geq 0,\\
& x = u & \text{and} & g(x) \leq 0,\\
\text{or} & l < x < u & \text{and} & g(x) = 0.
\end{array}
\]

These forms can be generalized by substituting a function $f(x)$ for the variable $x$.  Thus,
a \textit{generalized mixed complementarity condition} can be expressed as
\[
l \leq f(x) \leq u  \;\bot\;  g(x),
\]
which expresses the complementarity restriction that at least one of the following
must hold:
\[
\begin{array}{llll}
& f(x) = l & \text{and} & g(x) \geq 0,\\
& f(x) = u & \text{and} & g(x) \leq 0,\\
\text{or} & l < f(x) < u & \text{and} & g(x) = 0.
\end{array}
\]
For completeness, note that the complementarity condition
\[
f(x)  \;\bot\;  g(x) = 0
\]
is a special case where the function $f(x)$ is unbounded.


\subsection{Complementarity Expressions}

The design of complementarity conditions in Pyomo relies on the
specification of Pyomo constraint expressions.  A Pyomo constraint
expression defines an equality, a simple inequality, or a pair of
inequalities.  For example:
\[
\begin{array}{l}
{expr}_1 = {expr}_2\\
{expr}_1 <= {expr}_2\\
{const}_1 <= {expr}_2 <= {const}_2
\end{array}
\]
where ${const}_i$ are constant arithmetic expressions that may only
contain variables that are fixed, and ${expr}_i$ are arithmetic expressions
that contain unfixed variables.

A complementarity condition is defined with a pair of constraint expressions 
\[
{l}_1 <= {expr}_1 <= {u}_1 \;\;\bot\;\; {l}_2 <= {expr}_2 <= {u}_2,
\]
where exactly two of the constant bounds $l_1$, $u_1$, $l_2$ and
$u_2$ are finite.  The non-finite bounds values are omitted in
practice, so this condition directly describes a classical or mixed
complementarity condition.
Additionally, a complementarity condition can be expressed with a simple inequality:
\[
{l}_1 <= {expr}_1 <= {u}_1 \;\;\bot\;\; {expr}_2 <= {expr}_3.
\]
This complementarity condition is implicitly transformed to a form with constant bounds:
\[
{l}_1 <= {expr}_1 <= {u}_1 \;\;\bot\;\; {expr}_2 - {expr}_3 <= 0.
\]


\subsection{Modeling Mixed-Complementarity Conditions}

Pyomo employs an object-oriented strategy for representing models.
A Pyomo model object contains modeling components that define
standard elements of algebraic models (e.g. parameters, sets,
variables, constraints, and objectives).  This allows Pyomo to
automatically manage the naming of AML components, and multiple
Pyomo models can be simultaneously defined.

Additionally, Pyomo's modeling capabilities can be extended by
simply defining new modeling components.  Pyomo's \texttt{pyomo.mpec}
package defines the \texttt{Complementarity} component that is used
to declare complementarity conditions.

For example, consider the \texttt{ralph1} problem in MacMPEC~\cite{MacMPEC}:
\[
\begin{array}{ll}
\min & 2 x - y\\
     & 0 \leq y \;\bot\; y \geq x\\
     & x,y \geq 0
\end{array}
\]

\if 0
The following script defines a Pyomo model for \texttt{ralph1}:
\listing{examples/ralph1_strip.py}{}{1}{14}
The first lines in this script import Pyomo packages:
\listing{examples/ralph1.py}{imports}{4}{5}
The first line imports \code{pyomo.environ} to initialize Pyomo's environment, and it
imports the core modeling components from Pyomo.  The 
second line imports modeling components for complementarity
conditions.
The subsequent lines in this script create a model, declare variables
\code{x} and \code{y}, declare an objective \code{f1}, and
declare a complementarity condition \code{compl}.  
\fi

The complementarity condition is declared with the \code{Complementarity}
component.  In the simplest case, this Python class takes a keyword
argument \code{expr} that contains the value of the \code{complements}
function.  This function accepts two Pyomo constraint expressions
that are used to declare a complementarity condition.

\if 0
Pyomo also supports indexed components, where a set of components
are initialized over an index set using a construction rule.
Thus, the \texttt{Complementarity} component can be declared
with an index set.  For example, consider the following model, \code{indexed}:
\[
\begin{array}{lll}
\min & \sum_{i=1}^n i (x_i - 1)^2 &\\
     & 0 \leq x_i \;\bot\; 0 \leq x_{i+1} & \;\; i=1,\ldots,n-1
\end{array}
\]
The following script defines a Pyomo model for \code{indexed} with $n=5$:
\listing{examples/ex1a.py}{}{1}{16}
The complementarity conditions are defined with a single
\code{Complementarity} component that is indexed over the set $1,
\ldots, n-1$ and initialized with a construction rule \code{compl\_}.
This rule is a function that accepts a model instance and an index,
and returns the $i$-th complementarity
condition.

The declared set of indexes may be a superset of the indices that define
complementarity conditions.
For example, if the construction rule returns \code{Complementarity.Skip}, then the corresponding
index is skipped.  For example:
\listing{examples/ex1d.py}{}{1}{18}

This example can also be expressed with the 
\code{ComplementarityList} component:
\listing{examples/ex1b.py}{}{1}{18}
This component defines a list of complementarity conditions.  The list index can be 
used in Pyomo, but this component simplifies the declaration of models for which 
the index values are not important.  The \code{ComplementarityList} 
component can also be defined with a rule that iteratively returns complementarity conditions:
\listing{examples/ex1c.py}{}{1}{19}
%The value \code{ComplementarityList.End} is used to denote the end
%of the list.

Similarly, the construction rule may be a list expression that
generates a sequence of complementarity conditions:
\listing{examples/ex1e.py}{}{1}{15}
\fi

\SANDsection{MPEC Transformations}
\label{sec:transformations}

Pyomo's object-oriented design supports the structured transformation
of models.  Pyomo can iterate through model components as well as
nested model blocks.  Thus, model components can be easily transformed
locally, and global data can be collected to support global
transformations.  Further, Pyomo components and blocks can be
activated and deactivated, which facilitates \textit{in place}
transformations that do not require the creation of a separate
copy of the original model.

\if 0
Pyomo's \texttt{pyomo.mpec} package defines several model transformations
that can be easily applied.  For example, if \code{model} defines
an MPEC model (as in our previous examples), then the following
example illustrates how to apply a model transformation:
\listing{examples/ex2.py}{transform}{19}{19}
In this case, the \code{mpec.simple\_nonlinear} transformation is applied.
The following sections describe the transformations currently
supported in \code{pyomo.mpec}.
\fi

\subsection{Standard Form}

In Pyomo, a complementarity condition is expressed as a pair of
constraint expressions
\[
{l}_1 <= {expr}_1 <= {u}_1 \;\;\bot\;\; {l}_2 <= {expr}_2 <= {u}_2,
\]
where exactly two of the constant bounds $l_1$, $u_1$, $l_2$ and
$u_2$ are finite.  The non-finite bounds are typically omitted, but
the value \code{None} can be used to express infinite bounds.
Additionally, each constraint expression can be expressed with a
simple inequality of the form
\[
{expr}_1 <= {expr}_2.
\]

The \code{mpec.simple\_nonlinear} transformation reformulates each
complementarity condition in a model into a standard form:
\[
{l}_1 <= expr <= {u}_1 \;\;\bot\;\; {l}_2 <= var <= {u}_2,
\]
where exactly two of the constant bounds $l_1$, $u_1$, $l_2$ and
$u_2$ are finite, and either $l_2$ is zero or both $l_2$ or $u_2$ are finite.

Note that this transformation creates new variables and constraints
as part of this transformation.  For example, the complementarity
condition
\[
1 \leq x + y \;\;\bot\;\; 1 \leq 2x - y,
\]
get re-expressed as the following:
\[
\begin{array}{l}
1 \leq x + y\\
v = 2x - y - 1\\
v \in \mathbb{R}, v \ge 0
\end{array}
\]

For each complementary condition object, the new variable and
constraints are added as additional components within the
complementarity object.  Thus, the overall structure of the MPEC model
is not changed by this transformation.

\subsection{Simple Nonlinear}

The \code{mpec.simple\_nonlinear} transformation begins by applying
the \code{mpec.standard\_form} transformation.  Subsequently, a
nonlinear constraint is created that defines the complementarity
condition.  This is a simple nonlinear transformation adapted from
\citet{FerDirMee05}, which can be described by three different
cases:
\begin{itemize}

\item If $l_1$ is finite, then the following constraint is defined:
\[
    (expr - l_1) * v \leq \epsilon
\]

\item If $u_1$ is finite, then the following constraint is defined:
\[
    (u_1 - expr) * v \leq \epsilon
\]

\item If $l_2$ and $u_2$ are both finite, then the following constraints are defined:
\[
\begin{array}{l}
(var - l_2) * expr \leq \epsilon\\
(var - u_2) * expr \leq \epsilon\\
\end{array}
\]

\end{itemize}
Each of these cases ensure that the complementarity condition is
met when $\epsilon$ is zero.  For example, in the first case, we
know that $0 \leq v$ and $0 \leq expr - l_1$.  When $\epsilon$ is
zero, this constraint ensures that either $v$ is zero or $expr -
l_1$ is zero.

This transformation uses the parameter \code{mpec\_bound},
which defines the value for $\epsilon$ for every complementarity
condition.  This allows for the specification of a relaxed nonlinear
problem, which may be easier to optimize with some nonlinear
programming solvers.
The default value of \code{mpec\_bound} is zero.


\subsection{Simple Disjunction}

The \code{mpec.simple\_disjunction}
transformation expresses a complementarity condition as a disjunctive
program.  
We are given a complementarity condition defined with a pair of constraint expressions 
\[
{l}_1 <= {expr}_1 <= {u}_1 \;\;\bot\;\; {l}_2 <= {expr}_2 <= {u}_2,
\]
where exactly two of the constant bounds $l_1$, $u_1$, $l_2$ and
$u_2$ are finite.  Without loss of generality, we assume that either $l_1$ or $u_1$ is finite.

This transformation can be described by
three different cases:
\begin{itemize}

\item If the first constraint is an equality, then the complementarity condition is trivially replaced by that equality constraint.

\item If both bounds on the first constraint are finite but different, then the disjunction has the form:
\[
\left[\begin{array}{c}
Y_1\\
l_1 = {expr}_1\\
{expr}_2 \geq 0
\end{array}\right]
\bigvee
\left[\begin{array}{c}
Y_2\\
{expr}_1 = u_1\\
{expr}_2 \leq 0
\end{array}\right]
\bigvee
\left[\begin{array}{c}
Y_3\\
l_1 \leq {expr}_1 \leq u_1\\
{expr}_2 = 0
\end{array}\right]
\]\[
% You need a logic constraint so that only one of the disjuncts is
% selected.  I don't like this notation (and would rather express the
% XOR between the disjuncts, but this is the notation that Ignacio has
% consistently used.
Y_1 \veebar Y_2 \veebar Y_3 = True
\]
% WEH: This seems redundant
%\[
%Y_1 \land Y_2 \land Y_3 = False
%\]\[
% I am being pedantic here: you need to set the domain of the logical
% indicator variables.
\[
Y_1, Y_2, Y_3 \in \{True, False\}
\]

\item Otherwise, each constraint is a simple inequality.  The complementarity condition is reformulated as
\[
0 <= \overline{expr}_1  \;\;\bot\;\; 0 <= \overline{expr}_2,
\]
and the disjunction has the form:
\[
\left[\begin{array}{c}
Y\\
0 = \overline{expr}_1\\
0 \leq \overline{expr}_2
\end{array}\right]
\bigvee
\left[\begin{array}{c}
\lnot Y\\
0 \leq \overline{expr}_1\\
0 = \overline{expr}_2
\end{array}\right]
\]\[
Y \in \{ True, False \}
\]

\end{itemize}

This transformation makes use of modeling components and transformations
from Pyomo's \code{pyomo.gdp} package~\cite{PyomoGDP}.  The
transformation expresses each of the disjunctive terms explicitly using
\code{Disjunct} components and the \emph{select exactly one} logical
condition using the \code{Disjunction} component.  The transformation
adds the \code{Disjunct} and \code{Disjunction} components within the
objects that represent the complementarity conditions.  It then recasts
the modified complementarity components into simple \code{Block}
components.  This localizes all changes to the model to the individual
complementarity components.  Subsequent transformation of the
disjunctive expressions to algebraic constraints can be effected through
either Big-M (\code{gdp.bigm}) or Convex Hull (\code{gdp.chull})
transformations.



\subsection{AMPL Solver Interface}

Solvers like PATH~\cite{FerMun98} have been tailored to work with
the AMPL Solver Library (ASL).  AMPL uses \code{nl} files to
communicate with solvers, which read \code{nl} files with the ASL.
Pyomo can also create \code{nl} files, and the \code{mpec.nl}
transformation processes \code{Complementarity} components into a
canonical form that is suitable for this format~\cite{FerFouGay99}.


\SANDsection{Solver Interfaces and Meta-Solvers}
\label{sec:solvers}

Pyomo supports interfaces to third-party solvers as well as meta-solvers that 
apply transformations and third-party solvers, perhaps in an iterative manner.
The \code{pyomo.mpec} package includes an interface to the \code{PATH} solver, as well as several 
meta-solvers.  These are described in this section, and examples are provided that 
employ the \code{pyomo} command-line interface.


\subsection{Nonlinear Reformulations}

The \code{mpec.simple\_nonlinear} transformation provides a generic
way for transforming an MPEC into a nonlinear program.  When the
MPEC only has continuous decision variables, the resulting model
can be optimized by a wide range of solvers.

\if 0
For example, the \code{pyomo} command-line interface allows the
user to specify a nonlinear solver and a model transformation that
is applied to a model:
\listing{examples/nlp1.sh}{pyomo}{4}{4}
This example illustrates the use of the \code{ipopt} interior-point
solver with the\linebreak \code{mpec.simple\_nonlinear} transformation.  
When a transformation is used directly like this, the results
that are returned to the user include decision variables for the transformed
model.  Pyomo does not have general capabilites for mapping a
solution back into the space from the original model.  In this
example, the results object includes values for the $x$ variables
as well as the variables $v$ introduced when applying the transformation
to the standard form (see above).

Pyomo includes a meta-solver, \code{mpec\_nlp} that applies the
nonlinear transformation, performs optimization, and then returns
results for the original decision variables.  For example, \code{mpec\_nlp} 
executes the same logic as the previous \code{pyomo} example:
\listing{examples/nlp2.sh}{pyomo}{4}{4}
Additionally, this meta-solver can also manipulate the $\epsilon$
values in the model, starting with larger values and iteratively
tightening them to generate a more accurate model.
\listing{examples/nlp3.sh}{pyomo}{4}{5}
This approach may be useful when using a nonlinear solver that has difficulty
optimizing with equality constraints.
\fi


\subsection{Disjunctive Reformulations}

The \code{mpec.simple\_disjunction} transformation provides a generic
way for transforming an MPEC into a disjunctive program.  The
\code{mpec\_minlp} solver applies this transformation to create a
nonlinear disjunctive program, and then further reformulates the
disjunctive model using a ``Big-M'' transformation that is provided
by the \code{pyomo.gdp} package.  The resulting transformation is
similar the reformulation of bilevel models described by~\citet{ForMcC81}.
If the original model was nonlinear, then the resulting model is a
mixed-integer nonlinear program (MINLP).  Pyomo includes interfaces
to solvers that use the AMPL Solver Library (ASL), so \code{mpec\_minlp} can 
optimize nonlinear MPECs with a solver like Couenne~\cite{COUENNE}.

\if 0
If the original model was a linear MPEC, then the resulting model
is a mixed-integer linear program that can be globally optimized
(e.g. see~\citet{HuMitPanBenKun08,Jud11}).  For example, the
\code{pyomo} command can be used to execute the \code{mpec\_minlp}
solver using a specified MIP solver:
\listing{examples/mip1.sh}{pyomo}{4}{4}
Note that Pyomo includes interfaces to a variety of commonly used
MIP solvers, including CPLEX, Gurobi, CBC, and GLPK.
\fi


\subsection{PATH and the ASL Solver Interface}

Pyomo's solver interface for the AMPL Solver Library (ASL) applies
the \code{mpec.nl} transformation, writes an AMPL \code{.nl} file, executes
an ASL solver, and then loads the solution into the original model.
Pyomo provides a custom interface to the PATH solver~\cite{FerMun98},
which simply allows the solver to be specified as \code{path} while
the solver executable is named \code{pathamp}.

\if 0
The \code{pyomo} command can execute the PATH solver by simply specifying the 
\code{path} solver name.  For example, consider the \code{munson1} problem 
from MCPLIB:
\listing{examples/munson1.py}{}{1}{23}
%\pagebreak
This problem can be solved with the following command:
\listing{examples/path1.sh}{pyomo}{4}{4}
\fi



\SANDsection{Discussion}
\label{sec:discussion}

Pyomo supports the ability to model complementarity conditions in a manner that is
similar to other AMLs.  For example, Pyomo's \code{pyomo.data}
package~\cite{pyomo.data} includes Pyomo formulations for many of
the \mbox{MacMPEC}~\cite{MacMPEC} and MCPLIB~\cite{MCPLIB} models, which were originally formulated in GAMS and AMPL.  However,
Pyomo does not currently support related modeling capabilities for
equilibrium models, variational inequalities and embedded models,
which are supported by the GAMS extended mathematical programming
framework~\cite{FerDirJagMee09}.

The tranformations and meta-solvers currently included in Pyomo
illustrate how Pyomo's MPEC modeling capability can be leveraged.
We expect these capabilities to mature and expand in response to
application needs.  For example, the \code{mpec.simple\_nonlinear}
transformation could be expanded to support reformulations that
are well-suited for sequential quadratic programming solvers~\cite{Ley06}.
Similarly, current meta-solvers could be extended to directly
support the communication of suffix information from the solver
back to the original model.

